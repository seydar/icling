% FILE:		    RegularExpressionGrammar.tex
% PURPOSE:  	The theory and use of regular expression grammars
% COPYRIGHT:  W.M.McKeeman 2007


\input bookhead

\begin{document}

\begin{center}
\begin{small}
\noindent --contents of Regular Expression Grammars--
\begin{quote}
\raggedright
Definition\\
Self-describing Regular Expression Grammar\\
Rewriting Regular Expression Grammars\\
Fusion\\
Removal\\
Back-substituting\\
Regular Expression to \cfg \\
Using Regular Expressions\\
\end{quote}
--requires--
\begin{quote}
\raggedright
Context-free Grammars\\
Regular Expressions\\
Finite Automata\\
\end{quote}
\end{small}
\end{center}

\vspace{1em}


\noindent {\large \bf Regular Expression Grammars} 

\begin{quote}\raggedleft
                                          In a lot of mathematical work \\
                                      the use of an appropriate notation \\
                                                makes all the difference.\\
					--- {\em Edsger Dijkstra}
% ref ???
\end{quote}

\subsubsection{Definition of Regular Expression Grammars}}

Extending the \cfg\ to allow regular expressions of symbols 
and phrase names on the right hand side of reductions
makes the grammar more compact and often more readable;
it does not change the set of languages that can be defined.
The regular expression meta-operators for intersection and complement 
are excluded because they are somewhere between 
inefficient and impossible to implement.
The extension is called a regular expression grammar (\reg).

\subsubsection{Self-describing \reg}

\stepcounter{table}					% simulate \table
\newcounter{SelfReg}[chapter]
\setcounter{SelfReg}{\value{table}}

As in the self-describing \cfg, ignoring whitespace, and using 
grammatical builtins
{\tt Letter}, {\tt Digit}, and {\tt Any}

\begin{samepage}
\begin{tt}
\begin{tabbing}
xxxxxxxxxxx\=xxxxxxxx\=xxx\=\kill
\>Grammar \>=\>Rule*;                                           \\
\>Rule    \>=\>Name \sq=\sq\ RegExp \sq;\sq;                    \\
\>RegExp  \>=\>Alt* (\sq|\sq\ Alt*)*;                           \\
\>Alt     \>=\>Rep (\sq*\sq\ | \sq+\sq\ | \sq?\sq)?;            \\
\>Rep     \>=\>Name | Input | \sq(\sq\ RegExp \sq)\sq;          \\
\>Name    \>=\>Letter (Letter | Digit)*;                        \\
\>Input   \>=\>\sq\sq\sq\ Any \sq\sq\sq;                        \\
\end{tabbing}
\end{tt}
\begin{center}
Table~\thetable: A Self-describing \tname{reg}
\end{center}
\end{samepage}

\subsection{Rewriting Regular Expression Grammars}

\noindent Like \cfg{s}, \reg{s} can be rewritten without changing the
language. One use of the following rewriting rules is to transform any
\cfg\ into a \reg\ and {\em vice-versa}.

\vspace{1em}

\stepcounter{table}					% simulate \table
\newcounter{RegTx}[chapter]
\setcounter{RegTx}{\value{table}}

\begin{samepage}
\begin{tabbing}{lll}
xxxxxxxxxxx\=xxxxxxxxxxxxxxxxxxxxxxx\=\kill
\underline{\em transform}
  \>\quad\underline{\em condition} 
  \>\quad\underline{${\cal T}(\Pi)$}                       \\
substitute
  \> $A\la\alpha B\gamma\in\Pi\wedge B\la\beta\in\Pi$
  \> $\Pi \cup \{ A\la\alpha(\beta)\gamma \}$            \\
combine
  \> $A\la\alpha\in \Pi\wedge A\la\beta\in\Pi$
  \> $\Pi-\{A\la\alpha, A\la\beta\}\cup\{A\la\alpha|\beta\}$\\
left recursion
  \> $A\la\alpha\in \Pi\wedge A\la A \beta\in\Pi$
  \> $\Pi-\{A\la\alpha, A\la A\beta\}\cup\{A\la\alpha(\beta)^*\}$\\
right recursion
  \> $A\la\alpha\in \Pi\wedge A\la\beta A\in\Pi$
  \> $\Pi-\{A\la\alpha, A\la\beta A\}\cup\{A\la(\beta)^*\alpha\}$\\
meta-parens
  \> $A\la(\alpha)\in \Pi$
  \> $\Pi-\{A\la(\alpha)\}\cup\{A\la\alpha\}$                    \\
\end{tabbing}
\end{samepage}

\subsubsection{Substitute}

The substitution rule from {\cfg}s can be applied to {\reg}s 
so long as meta-parentheses are added.

\subsubsection{Combine (general form)}

The rules

$A\la\alpha_1$ \quad $A\la\alpha_1$ \ldots $A\la\alpha_m$

\noindent
can be replaced with a regular expression rule

$A\la \alpha_1|\alpha_2|\ldots|\alpha_m$

\subsubsection{Recursion}
Both of the recursion rules can also be applied in reverse to remove
the Kleene `*'.

\subsubsection{Left Recursion Removal}
The rules

$A\la\alpha_1$ \quad $A\la\alpha_2$; \ldots $A\la\alpha_m$

\noindent
and the rules

$A\la A\beta_1$; $A\la A\beta_2$; ... $A\la A\beta_n$

\noindent
can be turned into the single rule

$A\la$($\alpha_1|\alpha_2 ... |\alpha_m$)($\beta_1|\beta_2 ... |\beta_n$)*


\subsection{\reg\ to \cfg}

Given a regular expression $R$, the regular expression grammar (\reg)
\begin{displaymath}
\>G\la R
\end{displaymath}
describes the same language as the regular expression $R$.
Using the \reg\ transformations, the regular expression meta-symbols
can be systematically removed, resulting in a \cfg\ that 
still describes the same language.

Continuing, the \cfg\ can be transformed into a left-linear form
describing a \dfa.

The process can be reversed.
In the general case, starting from an arbitrary \cfg,
the reverse process will not end in a single rule.

\subsection{Using \reg{s}}

The \reg\ notation will be applied to the two main components of the 
compiler front end, parsing and scanning.  
Its obvious descriptive power is offset by not providing an obvious 
syntax tree.  Because of this we usually need both a \cfg\
and a \reg\ for each computer language.

\subsubsection{Exercise}
\begin{enumerate}\setcounter{enumi}{\value{RunningExercise}}

\item Describe the X language using an \reg.

\item What is the shortest self describing grammar 
you can devise?  You may want to exclude some of the meta-symbols.

\setcounter{RunningExercise}{\value{enumi}}
\end{enumerate}


\end{document}

