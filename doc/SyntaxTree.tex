% FILE:		    SyntaxTree.tex
% PURPOSE:  	Building a syntax tree
% COPYRIGHT:    W.M.McKeeman 2008


\input bookhead

\begin{document}

\begin{center}
\begin{small}
\noindent --contents of Syntax Tree--
\begin{quote}
\raggedright
Building Syntax Trees\\
An Example\\
Tree Transformations\\
Intermediate Representation\\
Abstract Syntax Tree\\
\end{quote}
--requires--
\begin{quote}
\raggedright
Notation Supporting Grammars\\
Context-free Grammars\\
Input-output Grammars
\end{quote}
\end{small}
\end{center}

\vspace{1em}


\noindent {\large \bf Building Syntax Trees} 
\vspace{1em}

\noindent Building a syntax tree from a shift/reduce sequence requires a 
simple algorithm:
\begin{itemize}
\item Stack each shifted token.
\item Build a node from the top of the stack to match each reduce.
\end{itemize}

\noindent Here is the \iog\ for {\em Proposition}.
It describes the constant formulas of the propositional calculus.
It has 10 rules.  
The rule numbers are used as output symbols.
The {\em chicken-foot\/} symbol $\perp$ is used for end-of-file.
\begin{em}
\begin{tabbing}
xxxxxxxxxx\=Propositionxx\=xxx\=xxxxxxxxxxxxxxxxxxxxxxxxxxxx\=\kill
\>Proposition \>\la\> Disjunction $\perp$             \>{\rm 0}    \\        
\>Disjunction \>\la\> Disjunction $\vee$ Conjunction  \>{\rm 1}    \\
\>Disjunction \>\la\> Conjunction                     \>{\rm 2}    \\
\>Conjunction \>\la\> Conjunction $\wedge$ Negation   \>{\rm 3}    \\
\>Conjunction \>\la\> Negation                        \>{\rm 4}    \\
\>Negation    \>\la\> $\neg$ Boolean                  \>{\rm 5}    \\
\>Negation    \>\la\> Boolean                         \>{\rm 6}    \\
\>Boolean     \>\la\> {\rm t}                         \>{\rm 7}    \\
\>Boolean     \>\la\> {\rm f}                         \>{\rm 8}    \\
\>Boolean     \>\la\> {\rm(} Disjunction {\rm)}       \>{\rm 9} 
\end{tabbing}
\end{em}

Applying the two-rule algorithm to the shift/reduce sequence
leads to a syntax tree.  
The nodes are labelled with the rule numbers.  
The input symbols are at the leaves.  

\vspace{1em}
\noindent {\bf An Example} 
\vspace{1em}

\noindent In the display of the steps below, the input text is on the right
and the successive states of the stack are on the left. 
The entries in the stack are tree nodes.
The tree is pushed out to the left (horizontal tree).

\vspace{1em}
\noindent The shift/reduce sequence for

    (f$\vee$t)$\perp$

\noindent is

    (f8642$\vee$t7641)9642$\perp$0

\pagebreak

\begin{picture}(500,25)(-50,0)
\put(90,0){\em treelets}
\put(140,0){\em stack}
\put(210,0){\em input}
\end{picture}

\begin{picture}(500,25)(-50,0)
\put(149,-4){\line(1,0){10}}  %
\put(149,-4){\line(0,1){12}}  %
\put(159,-4){\line(0,1){12}}  %
\put(180,0){(f8642$\vee$t7641)9642$\perp$0}
\end{picture}

\begin{picture}(500,25)(-50,0)
\put(152,0){(}
\put(149,-4){\line(1,0){10}}  %
\put(149,-4){\line(0,1){12}}  %
\put(159,-4){\line(0,1){12}}  %
\put(180,0){f8642$\vee$t7641)9642$\perp$0}
\end{picture}

\begin{picture}(500,35)(-50,0)
\put(152,0){(}
\put(152,10){f}
\put(149,-4){\line(1,0){10}}  %
\put(149,-4){\line(0,1){22}}  %
\put(159,-4){\line(0,1){22}}  %
\put(180,10){8642$\vee$t7641)9642$\perp$0}
\end{picture}

\begin{picture}(500,35)(-50,0)
\put(152,0){(}
\put(152,10){8}
\put(149,-4){\line(1,0){10}}  %
\put(149,-4){\line(0,1){22}}  %
\put(159,-4){\line(0,1){22}}  %
\put(146,13){\vector(-1,0){10}}
\put(128,10){f}
\put(180,10){642$\vee$t7641)9642$\perp$0}
\end{picture}

\begin{picture}(500,35)(-50,0)
\put(152,0){(}
\put(152,10){6}
\put(149,-4){\line(1,0){10}}  %
\put(149,-4){\line(0,1){22}}  %
\put(159,-4){\line(0,1){22}}  %
\put(146,13){\vector(-1,0){10}}
\put(128,10){8}
\put(124,13){\vector(-1,0){10}}
\put(106,10){f}
\put(180,10){42$\vee$t7641)9642$\perp$0}
\end{picture}

\begin{picture}(500,35)(-50,0)
\put(152,0){(}
\put(152,10){4}
\put(149,-4){\line(1,0){10}}  %
\put(149,-4){\line(0,1){22}}  %
\put(159,-4){\line(0,1){22}}  %
\put(146,13){\vector(-1,0){10}}
\put(128,10){6}
\put(124,13){\vector(-1,0){10}}
\put(106,10){8}
\put(102,13){\vector(-1,0){10}}
\put( 84,10){f}
\put(180,10){2$\vee$t7641)9642$\perp$0}
\end{picture}

\begin{picture}(500,35)(-50,0)
\put(152,0){(}
\put(152,10){2}
\put(149,-4){\line(1,0){10}}  %
\put(149,-4){\line(0,1){22}}  %
\put(159,-4){\line(0,1){22}}  %
\put(146,13){\vector(-1,0){10}}
\put(128,10){4}
\put(124,13){\vector(-1,0){10}}
\put(106,10){6}
\put(102,13){\vector(-1,0){10}}
\put( 84,10){8}
\put( 80,13){\vector(-1,0){10}}
\put( 62,10){f}
\put(180,10){$\vee$t7641)9642$\perp$0}
\end{picture}

\begin{picture}(500,45)(-50,0)
\put(152,0){(}
\put(152,10){2}
\put(151,20){$\vee$}
\put(149,-4){\line(1,0){10}}  %
\put(149,-4){\line(0,1){32}}  %
\put(159,-4){\line(0,1){32}}  %
\put(146,13){\vector(-1,0){10}}
\put(128,10){4}
\put(124,13){\vector(-1,0){10}}
\put(106,10){6}
\put(102,13){\vector(-1,0){10}}
\put( 84,10){8}
\put( 80,13){\vector(-1,0){10}}
\put( 62,10){f}
\put(180,20){t7641)9642$\perp$0}
\end{picture}

\begin{picture}(500,55)(-50,0)
\put(152,0){(}
\put(152,10){2}
\put(151,20){$\vee$}
\put(152,30){t}
\put(149,-4){\line(1,0){10}}  %
\put(149,-4){\line(0,1){42}}  %
\put(159,-4){\line(0,1){42}}  %
\put(146,13){\vector(-1,0){10}}
\put(128,10){4}
\put(124,13){\vector(-1,0){10}}
\put(106,10){6}
\put(102,13){\vector(-1,0){10}}
\put( 84,10){8}
\put( 80,13){\vector(-1,0){10}}
\put( 62,10){f}
\put(180,30){7641)9642$\perp$0}
\end{picture}

\begin{picture}(500,55)(-50,0)
\put(152,0){(}
\put(152,10){2}
\put(151,20){$\vee$}
\put(152,30){7}
\put(128,30){t}
\put(149,-4){\line(1,0){10}}  %
\put(149,-4){\line(0,1){42}}  %
\put(159,-4){\line(0,1){42}}  %
\put(146,33){\vector(-1,0){10}}
\put(146,13){\vector(-1,0){10}}
\put(128,10){4}
\put(124,13){\vector(-1,0){10}}
\put(106,10){6}
\put(102,13){\vector(-1,0){10}}
\put( 84,10){8}
\put( 80,13){\vector(-1,0){10}}
\put( 62,10){f}
\put(180,30){641)9642$\perp$0}
\end{picture}

\begin{picture}(500,55)(-50,0)
\put(152,0){(}
\put(152,10){2}
\put(151,20){$\vee$}
\put(152,30){6}
\put(149,-4){\line(1,0){10}}  %
\put(149,-4){\line(0,1){42}}  %
\put(159,-4){\line(0,1){42}}  %
\put(124,33){\vector(-1,0){10}}
\put(128,30){7}
\put(146,33){\vector(-1,0){10}}
\put(106,30){t}
\put(146,13){\vector(-1,0){10}}
\put(128,10){4}
\put(124,13){\vector(-1,0){10}}
\put(106,10){6}
\put(102,13){\vector(-1,0){10}}
\put( 84,10){8}
\put( 80,13){\vector(-1,0){10}}
\put( 62,10){f}
\put(180,30){41)9642$\perp$0}
\end{picture}

\begin{picture}(500,55)(-50,0)
\put(152,0){(}
\put(152,10){2}
\put(151,20){$\vee$}
\put(152,30){4}
\put(149,-4){\line(1,0){10}}  %
\put(149,-4){\line(0,1){42}}  %
\put(159,-4){\line(0,1){42}}  %
\put(146,33){\vector(-1,0){10}}
\put(128,30){6}
\put(124,33){\vector(-1,0){10}}
\put(106,30){7}
\put(102,33){\vector(-1,0){10}}
\put( 84,30){t}
\put(146,13){\vector(-1,0){10}}
\put(128,10){4}
\put(124,13){\vector(-1,0){10}}
\put(106,10){6}
\put(102,13){\vector(-1,0){10}}
\put( 84,10){8}
\put( 80,13){\vector(-1,0){10}}
\put( 62,10){f}
\put(180,30){1)9642$\perp$0}
\end{picture}

\begin{picture}(500,45)(-50,0)
\put(152,10){(}
\put(152,20){1}
\put(149,6){\line(1,0){10}}  %
\put(149,6){\line(0,1){22}}  %
\put(159,6){\line(0,1){22}}  %
\put(146,23){\vector(-1,-1){10}}
\put(146,23){\vector(-1,0){10}}
\put(146,23){\vector(-1,1){10}}
\put(128,10){2}
\put(128,20){$\vee$}
\put(128,30){4}
\put(124,33){\vector(-1,0){10}}
\put(106,30){6}
\put(100,33){\vector(-1,0){10}}
\put( 84,30){7}
\put( 80,33){\vector(-1,0){10}}
\put( 62,30){t}
\put(124,13){\vector(-1,0){10}}
\put(106,10){4}
\put(102,13){\vector(-1,0){10}}
\put( 84,10){6}
\put( 80,13){\vector(-1,0){10}}
\put( 62,10){8}
\put( 58,13){\vector(-1,0){10}}
\put( 40,10){f}
\put(180,20){)9642$\perp$0}
\end{picture}

\begin{picture}(500,45)(-50,0)
\put(152,10){(}
\put(152,20){1}
\put(152,30){)}
\put(149,6){\line(1,0){10}}  %
\put(149,6){\line(0,1){32}}  %
\put(159,6){\line(0,1){32}}  %
\put(146,23){\vector(-1,-1){10}}
\put(146,23){\vector(-1,0){10}}
\put(146,23){\vector(-1,1){10}}
\put(128,10){2}
\put(128,20){$\vee$}
\put(128,30){4}
\put(124,33){\vector(-1,0){10}}
\put(106,30){6}
\put(100,33){\vector(-1,0){10}}
\put( 84,30){7}
\put( 80,33){\vector(-1,0){10}}
\put( 62,30){t}
\put(124,13){\vector(-1,0){10}}
\put(106,10){4}
\put(102,13){\vector(-1,0){10}}
\put( 84,10){6}
\put( 80,13){\vector(-1,0){10}}
\put( 62,10){8}
\put( 58,13){\vector(-1,0){10}}
\put( 40,10){f}
\put(180,30){9642$\perp$0}
\end{picture}

\begin{picture}(500,45)(-50,0)
\put(151,20){9}
\put(149,16){\line(1,0){10}}  %
\put(149,16){\line(0,1){12}}  %
\put(159,16){\line(0,1){12}}  %
\put(146,23){\vector(-1,-1){10}}
\put(146,23){\vector(-1,0){10}}
\put(146,23){\vector(-1,1){10}}
\put(128,10){(}
\put(128,20){1}
\put(128,30){)}
\put(106,10){2}
\put(124,23){\vector(-1,-1){10}}
\put(124,23){\vector(-1,0){10}}
\put(124,23){\vector(-1,1){10}}
\put(106,20){$\vee$}
\put(106,30){4}
\put(102,33){\vector(-1,0){10}}
\put( 84,30){6}
\put( 78,33){\vector(-1,0){10}}
\put( 62,30){7}
\put( 58,33){\vector(-1,0){10}}
\put( 40,30){t}
\put(102,13){\vector(-1,0){10}}
\put( 84,10){4}
\put( 80,13){\vector(-1,0){10}}
\put( 62,10){6}
\put( 58,13){\vector(-1,0){10}}
\put( 40,10){8}
\put( 36,13){\vector(-1,0){10}}
\put( 18,10){f}
\put(180,20){642$\perp$0}
\end{picture}

\begin{picture}(500,45)(0,0)
\put(201,20){6}
\put(199,16){\line(1,0){10}}  %
\put(199,16){\line(0,1){12}}  %
\put(209,16){\line(0,1){12}}  %
\put(196,23){\vector(-1,0){10}}
\put(178,20){9}
\put(174,23){\vector(-1,-1){10}}
\put(174,23){\vector(-1,0){10}}
\put(174,23){\vector(-1,1){10}}
\put(156,10){(}
\put(156,20){1}
\put(156,30){)}
\put(134,10){2}
\put(152,23){\vector(-1,-1){10}}
\put(152,23){\vector(-1,0){10}}
\put(152,23){\vector(-1,1){10}}
\put(134,20){$\vee$}
\put(134,30){4}
\put(130,33){\vector(-1,0){10}}
\put(112,30){6}
\put(108,33){\vector(-1,0){10}}
\put( 90,30){7}
\put( 86,33){\vector(-1,0){10}}
\put( 68,30){t}
\put(130,13){\vector(-1,0){10}}
\put(112,10){4}
\put(108,13){\vector(-1,0){10}}
\put( 90,10){6}
\put( 86,13){\vector(-1,0){10}}
\put( 68,10){8}
\put( 64,13){\vector(-1,0){10}}
\put( 46,10){f}
\put(230,20){42$\perp$0}
\end{picture}

\begin{picture}(500,45)(0,0)
\put(201,20){4}
\put(199,16){\line(1,0){10}}  %
\put(199,16){\line(0,1){12}}  %
\put(209,16){\line(0,1){12}}  %
\put(196,23){\vector(-1,0){10}}
\put(178,20){6}
\put(174,23){\vector(-1,0){10}}
\put(156,20){9}
\put(152,23){\vector(-1,-1){10}}
\put(152,23){\vector(-1,0){10}}
\put(152,23){\vector(-1,1){10}}
\put(134,10){(}
\put(134,20){1}
\put(134,30){)}
\put(112,10){2}
\put(130,23){\vector(-1,-1){10}}
\put(130,23){\vector(-1,0){10}}
\put(130,23){\vector(-1,1){10}}
\put(112,20){$\vee$}
\put(112,30){4}
\put(108,33){\vector(-1,0){10}}
\put( 90,30){6}
\put( 86,33){\vector(-1,0){10}}
\put( 68,30){7}
\put( 64,33){\vector(-1,0){10}}
\put( 46,30){t}
\put(108,13){\vector(-1,0){10}}
\put( 90,10){4}
\put( 86,13){\vector(-1,0){10}}
\put( 68,10){6}
\put( 64,13){\vector(-1,0){10}}
\put( 46,10){8}
\put( 42,13){\vector(-1,0){10}}
\put( 24,10){f}
\put(230,20){2$\perp$0}
\end{picture}

\begin{picture}(500,45)(0,0)
\put(201,20){2}
\put(199,16){\line(1,0){10}}  %
\put(199,16){\line(0,1){12}}  %
\put(209,16){\line(0,1){12}}  %
\put(196,23){\vector(-1,0){10}}
\put(178,20){4}
\put(174,23){\vector(-1,0){10}}
\put(156,20){6}
\put(152,23){\vector(-1,0){10}}
\put(134,20){9}
\put(130,23){\vector(-1,-1){10}}
\put(130,23){\vector(-1,0){10}}
\put(130,23){\vector(-1,1){10}}
\put(112,10){(}
\put(112,20){1}
\put(112,30){)}
\put( 90,10){2}
\put(108,23){\vector(-1,-1){10}}
\put(108,23){\vector(-1,0){10}}
\put(108,23){\vector(-1,1){10}}
\put( 90,20){$\vee$}
\put( 90,30){4}
\put( 86,33){\vector(-1,0){10}}
\put( 68,30){6}
\put( 64,33){\vector(-1,0){10}}
\put( 46,30){7}
\put( 42,33){\vector(-1,0){10}}
\put( 24,30){t}
\put( 86,13){\vector(-1,0){10}}
\put( 68,10){4}
\put( 64,13){\vector(-1,0){10}}
\put( 46,10){6}
\put( 42,13){\vector(-1,0){10}}
\put( 24,10){8}
\put( 20,13){\vector(-1,0){10}}
\put(  2,10){f}
\put(230,20){$\perp$0}
\end{picture}

\begin{picture}(500,45)(0,0)
\put(200,30){$\perp$}
\put(201,20){2}
\put(199,16){\line(1,0){10}}  %
\put(199,16){\line(0,1){22}}  %
\put(209,16){\line(0,1){22}}  %
\put(196,23){\vector(-1,0){10}}
\put(178,20){4}
\put(174,23){\vector(-1,0){10}}
\put(156,20){6}
\put(152,23){\vector(-1,0){10}}
\put(134,20){9}
\put(130,23){\vector(-1,-1){10}}
\put(130,23){\vector(-1,0){10}}
\put(130,23){\vector(-1,1){10}}
\put(112,10){(}
\put(112,20){1}
\put(112,30){)}
\put( 90,10){2}
\put(108,23){\vector(-1,-1){10}}
\put(108,23){\vector(-1,0){10}}
\put(108,23){\vector(-1,1){10}}
\put( 90,20){$\vee$}
\put( 90,30){4}
\put( 86,33){\vector(-1,0){10}}
\put( 68,30){6}
\put( 64,33){\vector(-1,0){10}}
\put( 46,30){7}
\put( 42,33){\vector(-1,0){10}}
\put( 24,30){t}
\put( 86,13){\vector(-1,0){10}}
\put( 68,10){4}
\put( 64,13){\vector(-1,0){10}}
\put( 46,10){6}
\put( 42,13){\vector(-1,0){10}}
\put( 24,10){8}
\put( 20,13){\vector(-1,0){10}}
\put(  2,10){f}
\put(230,30){0}
\end{picture}

\begin{picture}(500,45)(0,0)
\put(226,25){0}
\put(224,22){\line(1,0){10}}  %
\put(224,22){\line(0,1){12}}  %
\put(234,22){\line(0,1){12}}  %
\put(220,27){\vector(-2,-1){10}}
\put(220,27){\vector(-2,1){10}}
\put(198,30){$\perp$}
\put(200,20){2}
\put(196,23){\vector(-1,0){10}}
\put(178,20){4}
\put(174,23){\vector(-1,0){10}}
\put(156,20){6}
\put(152,23){\vector(-1,0){10}}
\put(134,20){9}
\put(130,23){\vector(-1,-1){10}}
\put(130,23){\vector(-1,0){10}}
\put(130,23){\vector(-1,1){10}}
\put(112,10){(}
\put(112,20){1}
\put(112,30){)}
\put( 90,10){2}
\put(108,23){\vector(-1,-1){10}}
\put(108,23){\vector(-1,0){10}}
\put(108,23){\vector(-1,1){10}}
\put( 90,20){$\vee$}
\put( 90,30){4}
\put( 86,33){\vector(-1,0){10}}
\put( 68,30){6}
\put( 64,33){\vector(-1,0){10}}
\put( 46,30){7}
\put( 42,33){\vector(-1,0){10}}
\put( 24,30){t}
\put( 86,13){\vector(-1,0){10}}
\put( 68,10){4}
\put( 64,13){\vector(-1,0){10}}
\put( 46,10){6}
\put( 42,13){\vector(-1,0){10}}
\put( 24,10){8}
\put( 20,13){\vector(-1,0){10}}
\put(  2,10){f}
\end{picture}

\vspace{1em}
\noindent {\bf Tree Transformations}
\vspace{1em}

\noindent While xcom treats trees as read-only,
there is a whole class of useful transformations that can be applied
to trees to improve the quality of the resulting compiled code.
Some of these are described under the heading of {\bf Optimization}.

Tree transformations lead to the creation of new nodes and the 
abandonment of others.   
The tree mechanism needs to support these dynamic demands.
Some tree transformations lead to trees that could not represent
a program as written.  
This is ok until a diagnostic needs to be issued
for a tree construct that has no image in the source text.
All of these complexities and more have been solved in most
modern compilers.  


\vspace{1em}
\noindent {\bf Intermediate Representation}
\vspace{1em}

\noindent The result of transformations often takes the data structure
far from its origins as a syntax tree.  
In many compilers the information passed from front-end to back-end
is called the {\em intermediate representation} or IR.

\vspace{1em}

\noindent {\bf Abstract Syntax Tree}
\vspace{1em}

\noindent Most of the syntax tree is occupied by the empty transitions from 
one phrase name to the next.
The value of this form of grammar is that it expresses
associativity and operator precedence.
The existence of long chains of steps which, in themselves, 
convey no meaning, is a disadvantage of using unmodified syntax trees.
Compiler writing tools, generally speaking, eliminate the useless chains
by one means or another to give an Abstract Syntax Tree.  
Here is what the tree would look like for the example.

\begin{picture}(500,45)(100,0)
\put(224,25){0}
\put(220,27){\vector(-2,-1){10}}
\put(220,27){\vector(-2,1){10}}
\put(198,30){$\perp$}
\put(200,20){9}
\put(196,23){\vector(-1,-1){10}}
\put(196,23){\vector(-1,0){10}}
\put(196,23){\vector(-1,1){10}}
\put(178,10){(}
\put(178,20){1}
\put(178,30){)}
\put(156,10){t}
\put(174,23){\vector(-1,-1){10}}
\put(174,23){\vector(-1,0){10}}
\put(174,23){\vector(-1,1){10}}
\put(156,20){$\vee$}
\put(156,30){f}
\end{picture}

\noindent If there is no semantics associated with parentheses, 
they too can be eliminated

\begin{picture}(500,45)(100,0)
\put(224,25){0}
\put(220,27){\vector(-2,-1){10}}
\put(220,27){\vector(-2,1){10}}
\put(198,30){$\perp$}
\put(200,20){1}
\put(178,10){t}
\put(196,23){\vector(-1,-1){10}}
\put(196,23){\vector(-1,0){10}}
\put(196,23){\vector(-1,1){10}}
\put(178,20){$\vee$}
\put(178,30){f}
\end{picture}

\noindent And, finally, the operators are implicit in the rules, 
so they too can be eliminated.

\begin{picture}(500,45)(100,0)
\put(224,20){0}
\put(220,23){\vector(-1,0){10}}
\put(200,20){1}
\put(178,10){t}
\put(196,23){\vector(-1,-1){10}}
\put(196,23){\vector(-1,1){10}}
\put(178,30){f}
\end{picture}

\noindent The resulting tree is (obviously) much more concise and still contains
the information needed to compile code.  
One disadvantage in the last step of eliminating the operator
symbols is that their position in the source text goes with them.
This somewhat complicates the formulation of diagnostics involving
the operators.


\end{document}