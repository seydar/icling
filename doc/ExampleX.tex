% FILE:		    ExampleX.tex
% PURPOSE:  	Simple Examples of X
% COPYRIGHT: W.M.McKeeman 2007.  You may do anything you like with 
%    this file except remove or modify this copyright.
% MODS:      McKeeman - 2007.07.06 - original

\input xhead

\begin{document}

\begin{center}
\begin{small}
\noindent --contents of X Examples--
\begin{quote}
\raggedright
Example -- Primes\\
Example -- $\pi$\\
Example -- $e$\\
\end{quote}
\end{center}
\vspace{1em}
\end{center}

\begin{center} 
             {\large \bf X Examples} 
\end{center}
\vspace{1em}

\subsection{Example -- primes}

Suppose you wish to test whether an integer is prime.  
A simple approach is to examine the remainder after division by each 
feasible divisor.  One can quit after trying everything up to the square 
root of the candidate value.  Since fixed size 2's complement arithmetic
is used, this approach will eventually fail for large integers.

\begin{tt}
\begin{tabbing}
xxxx\=xx\=xxxxxxxxxxxxxxxxxxxxxxxxxxxxxxxxxxx\=\kill
\>n := PrimeCandidate;            \>     \>\bq\ input integer     \\
\>trial := 2;                     \>     \>\bq\ smallest possible \\
\>do trial*trial < n \& n//trial \tw= 0  ?                        \\
\>\> trial := trial + 1          \>\bq\ keep trying               \\
\>od;                                                             \\
\>NumberIsPrime := trial*trial > n   \>  \>\bq\ output
\end{tabbing}
\end{tt} 

\noindent In this case the input/output might appear as:
\begin{tt}
\begin{tabbing}
xxxx\=xx\=xxxxxxxxxxxxxxxxxxxxxxxxxxxxxxxxx\=\kill
\>   PrimeCandidate := 97;               \\
\>   NumberIsPrime := true;
\end{tabbing}
\end{tt}

\noindent The publication form of the prime program is 
\begin{tabbing} 
xxxx\=xxx\=xxxxxxxxxxxxxxxxxxxxxxxxxxxxxxxxxxxxxxxxxxxxxxxxxx\=\kill
\>n\ $\leftarrow$\ PrimeCandidate;   \>\>\bq\em\ input integer \\
\>trial\ $\leftarrow$\ 2;            \>\>\bq\em\ smallest possible \\
\>{\bf do} trial$\times$trial\ $<$\ n\ $\wedge$\ n$/\!/$trial\ $\ne$\ 0 $\Rightarrow$\
trial\ $\leftarrow$\ trial$+$1    \>\>\bq\em\ keep trying\\
\>{\bf od};\\
\>NumberIsPrime\ $\leftarrow$\ trial$\times$trial\ $>$\ n \>\>\bq\em\ output 
\end{tabbing}

\subsection{Example -- $\pi$}

\noindent
A simple (and inefficient) way to compute $\pi$ is to sum the series

$4*(1/1 - 1/3 + 1/5 - 1/7\ldots)$

\noindent
It turns out that the average of the last two partial sums is more accurate 
than the last sum by itself.  
On a machine using 32-bit \tname{IEEE} floating point,
one should not expect more than about 7 decimal digits accuracy.

\begin{tt}
\begin{tabbing}
xxxx\=xx\=xx\=xxxxxxxxxxxxxxxxxxxxxxxxxxxxxxx\=\kill
\>  old, sgn, step, lim  := 0.0, 1.0, 1, 1000;              \\
\>  do  step < lim ?                                        \\
\>\>      new := old+sgn/i2r(step);                         \\
\>\>      step := step+2;                                   \\
\>\>      sgn := -sgn                                       \\
\>\>      pi := 2.0*(old+new);        \>\>\bq\ result       \\
\>\>      old := new;                                       \\
\>  od
\end{tabbing}
\end{tt}

\noindent
The (perhaps surprisingly inaccurate) output is \begin{tt} 
\begin{tabbing}
xxxx\=xx\=xxxxxxxxxxxxxxxxxxxxxxxxxxxxxxxxx\=\kill
\>pi := 3.1415913;
\end{tabbing}
\end{tt}

\noindent Given suitable choices in the typesetting options, the typeset 
version might look like:

\begin{tabbing}
xxxx\=xx\=xx\=xxxxxxxxxxxxxxxxxxxxxxxxxxxxxxx\=\kill
\>  old, sgn, step, lim\ $\leftarrow$\ 0.0, 1.0, 1, 1000;              \\
\>  {\bf do}  step $<$ lim\ $\Rightarrow$                              \\
\>\>      new\ $\leftarrow$\ old$+$sgn$/$i2r(step);                    \\
\>\>      step\ $\leftarrow$\ step$+$2;                                \\
\>\>      sgn\ $\leftarrow$\ $-$sgn;                                   \\
\>\>      $\pi$\ $\leftarrow$\ 2.0$\times$(old$+$new);\>\>\bq\ result  \\
\>\>      old\ $\leftarrow$\ new;                                      \\
\>  {\bf od}
\end{tabbing}

\subsection{Example -- $e$}

\noindent
The base of natural logarithms can be expressed as a continued fraction:

\mbox{$2\ 1\ 2\ 1\ 1\ 4\ 1\ 1\ 6\ 1\ 1\dots$}.  

\noindent
Because the pattern is regular after the first two items {\tt 2\ 1}, 
the rest can be generated on the fly.  
Here is a program that does it.
\footnote{See Knuth, The Art of Computer Programming, volume 2 for more 
information.}

\begin{tt}
\begin{tabbing}
xxxx\=xx\=xxxxxxxxxxxxxxxxxxxxxxxxxxxxxxx\=\kill
\>  t := tol+0.0;         \>\>\bq\ input                            \\
\>  i := 2.0;             \>\>\bq\ start at 2 1 1 4 1 1 ...         \\
\>  ao, a := 2.0, 3.0;                    \>\>\bq\ numerators       \\
\>  bo, b := 1.0, 1.0;                    \>\>\bq\ denominators     \\
\>  diff := ao/bo - a/b;                                            \\
\>  do diff*diff < t*t ?                                           \\
\>\>      ao, a := a, ao+a*i;                \>\bq\ i               \\
\>\>      bo, b := b, bo+b*i;                                       \\
\>\>      ao, a := a, ao+a;                  \>\bq\ 1               \\
\>\>      bo, b := b, bo+b;                                         \\
\>\>      ao, a := a, ao+a;                  \>\bq\ 1               \\
\>\>      bo, b := b, bo+b;                                         \\
\>\>      i := i+2.0                                                \\
\>\>      diff := ao/bo - a/b;                                      \\
\>  od;                                                             \\
                                                                    \\
\>  e := a/b                             \>\>\bq\ output            \\
\end{tabbing}
\end{tt}

\end{document}