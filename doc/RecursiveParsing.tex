% FILE:		    RecursiveParsing.tex
% PURPOSE:  	The theory and how-to of top-down parsing
% COPYRIGHT:  W.M.McKeeman 2007

\input bookhead
\begin{document}

\begin{center}
\begin{small}

\noindent --contents of Recursive Parsing--
\begin{quote}
\raggedright
The {\em Proposition} example again\\
left recursion removal\\
recursive parser construction\\
MATLAB parser for {\em Proposition}\\
\end{quote}

\noindent --requires--
\begin{quote}
\raggedright
  Context-free Grammars\\
  Regular Expression Grammars\\
  Input-output Grammars
\end{quote}

\end{small}
\end{center}

\vspace{1em}

\noindent {\large \bf Recursive Parsing}

\vspace{1em}

\noindent Recalling the original \cfg\ for {\em Proposition\/} from
the section on context-free grammars

\begin{samepage}
\begin{em}
\begin{tabbing}
xxxxxxxxxx\=Propositionxx\=xxx\=\kill
\>Proposition \>\la\> Disjunction {\rm eof}              \\        
\>Disjunction \>\la\> Disjunction $\vee$ Conjunction     \\
\>Disjunction \>\la\> Conjunction                        \\
\>Conjunction \>\la\> Conjunction $\wedge$ Negation      \\
\>Conjunction \>\la\> Negation                           \\
\>Negation    \>\la\> $\neg$ Boolean                     \\
\>Negation    \>\la\> Boolean                            \\
\>Boolean     \>\la\> {\rm t}                            \\
\>Boolean     \>\la\> {\rm f}                            \\
\>Boolean     \>\la\> {\rm(} Disjunction {\rm)}
\end{tabbing}
\end{em}
\end{samepage}

\noindent
one can append the rule numbers as output symbols.  
In this case there is no difficulty distinguishing $V_O$
since the digits $0\ldots 9$ are not in $V_I$.

\begin{samepage}
\begin{em}
\begin{tabbing}
xxxxxxxxxx\=Propositionxx\=xxx\=xxxxxxxxxxxxxxxxxxxxxxxxxxxx\=\kill
\>Proposition \>\la\> Disjunction {\rm eof}           \>{\rm 0}    \\        
\>Disjunction \>\la\> Disjunction $\vee$ Conjunction  \>{\rm 1}    \\
\>Disjunction \>\la\> Conjunction                     \>{\rm 2}    \\
\>Conjunction \>\la\> Conjunction $\wedge$ Negation   \>{\rm 3}    \\
\>Conjunction \>\la\> Negation                        \>{\rm 4}    \\
\>Negation    \>\la\> $\neg$ Boolean                  \>{\rm 5}    \\
\>Negation    \>\la\> Boolean                         \>{\rm 6}    \\
\>Boolean     \>\la\> {\rm t}                         \>{\rm 7}    \\
\>Boolean     \>\la\> {\rm f}                         \>{\rm 8}    \\
\>Boolean     \>\la\> {\rm(} Disjunction {\rm)}       \>{\rm 9} 
\end{tabbing}
\end{em}
\end{samepage}

\noindent
Apply the transformation for \cfg\ left recursion removal,
\begin{tabbing}{ll}
xxxxxxxxx\=xxxxxxxxxxxxxxxxxxxxxxxxxxxxxxxxxxx\=\kill
given:
  \> $A\la\alpha\in \Pi\wedge A\la A \beta\in\Pi$ \\
add:
  \> $A\la\alpha(\beta)^*$\\
remove:
  \> $A\la\alpha, A\la A\beta$
\end{tabbing}

dragging the output symbols along with the rest.

\begin{samepage}
\begin{em}
\begin{tabbing}
xxxxxxxxxx\=Propositionxx\=xxx\=\kill
\>Proposition \>\la\> Disjunction {\rm eof} {\rm 0}                      \\        
\>Disjunction \>\la\> Conjunction {\rm 2} ($\vee$ Conjunction {\rm 1})*  \\
\>Conjunction \>\la\> Negation {\rm 4} ($\wedge$ Negation {\rm 3})*      \\
\>Negation    \>\la\> $\neg$ Boolean {\rm 5} $|$ Boolean {\rm 6}         \\
\>Boolean     \>\la\> {\rm t} {\rm 7} $|$ {\rm f} {\rm 8} $|$ {\rm(} Disjunction {\rm)} {\rm 9} 
\end{tabbing}
\end{em}
\end{samepage}

\noindent 
The above grammar is a template for a recursive parser 
for {\em Proposition}.  
Each phrase name gives rise to a corresponding function.  
Each such function relies on the functions it calls to ``do their job.''  
There is a variable {\tt next} which has the pending input symbol.  
There is a function {\tt scan} which steps ahead in the input.  
There is a function {\tt shift} which
reports that an input symbol has been shifted and calls {\tt scan}.
There is a function {\tt reduce} that reports that a grammar rule 
has been applied.
  
Here is the code in MATLAB, 
with \verb+~+ for $\neg$, \verb+|+ for $\vee$ and \verb+&+ for $\wedge$.

\vspace{1em}
\hrule
\begin{verbatim}
function sr = Proposition(src)

  EOF  = 0;  src = [src EOF];   % append artificial EOF
  next = ''; ip  = 1; scan();   % initialize next
  sr   = ''; op  = 1;           % next avail output position
    
  Disjunction();
  
  switch next
    case EOF; reduce('0');
    otherwise; error('missing EOF'); 
  end
  
  return;              % end of main function
  
  % ------------- nested functions ------------
  function Disjunction()
    Conjunction(); reduce('2');
    while next == '|'
      shift(); Conjunction(); reduce('1');
    end
  end

  function Conjunction()
    Negation(); reduce('4');
    while next == '&'
      shift(); Negation(); reduce('3');
    end
  end

  function Negation()
    switch next
      case '~'; shift(); Boolean(); reduce('5');
      otherwise; Boolean(); reduce('6');
    end
  end

  function Boolean()
    switch next
      case 't'; shift(); reduce('7');
      case 'f'; shift(); reduce('8');
      case '('; shift(); Disjunction();
        switch next
          case ')'; shift(); reduce('9');
          otherwise; error('missing )');
        end
      otherwise; error(['unexpected operand ' next]);
    end
  end

  function shift()
    emit(next); scan();
  end

  function reduce(r)
    emit(r);
  end

  function scan()
    next = src(ip); ip = ip+1;
  end

  function emit(s)
    sr(op) = s; op = op+1;
  end
end
\end{verbatim}
\hrule
\vspace{1em}

\noindent The result of \verb+Proposition('(f|t)')+
is \verb+(f8642|t7641)96420+
\vspace{1em}
\noindent To summarize, two shifts, four reduces, two more shifts, 
four more reduces, another shift, 5 reduces.

As it turns out, the  
parenthesis-free notation (PFN or reverse Polish) of Lukasiewicz 
is embedded in the reduce sequence.  If each input symbol consumed by 
a reduction is printed under the rule that consumed it we get

\begin{verbatim}
    8  6  4  2  7  6  4  1  9  6  4  2  0
    f           t        |  ()          eof
\end{verbatim}
\noindent discarding the parentheses (which is the point, after all)
only the \verb+ft|+ symbols and the eof remain.  This small fact is the
basis for compilers that go directly from the shift/reduce sequence to 
executable code.

\subsubsection{Exercise} Why does the PFN show up in the reduce sequence?

\end{document}

