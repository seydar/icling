% FILE:       FiniteAutomata.tex
% PURPOSE:    The definition of finite automata
% COPYRIGHT:  W.M.McKeeman 2007


\input bookhead

\begin{document}

\begin{center}
\begin{small}
\noindent --contents of Finite Automata--
\begin{quote}
\raggedright
Definitions\\
State-transition Diagram\\
\cfg\ for Automata\\
Non-deterministic to Deterministic\\
\end{quote}
--requires--
\begin{quote}
\raggedright
Notation Supporting Grammars\\
Context-free Grammars\\
\end{quote}
\end{small}
\end{center}

\vspace{1em}


\noindent {\large \bf Finite Automata} 
\vspace{1em}

Finite automata (\fa)\index{\fa} are abstract algorithms for 
the recognition of sequences.  
They are closely related to regular expressions and \cfg{s}.

Automata are also called state-transition machines.  
The central idea is that an automaton, at any one time, 
is in a unique state and can transition to some other state 
by reading input.  
The transitions are defined by a relation from state-input pairs to states.  
Transitions on the null string, i.e.~non-reading transitions, are allowed.  
The sequence of input values read is the string that is recognized. 

Processing begins in a {\em start state}.  
At each step the automaton examines the text, 
and based on its state-transition table, goes to another state.  
Each time an automaton transitions on an input symbol,
that symbol is discarded so that the next symbol may be processed.
Whenever the automaton is in a designated {\em final
state},\index{state, final} the input read so far is said to have been
{\em accepted}.\index{accepted} 
If, on the other hand, input appears 
for which there is no defined transition, 
the automaton is said to {\em reject\/} \index{reject} the input.  
Processing continues until the input is rejected or there is no more input.  

Finite automata are either deterministic \index{automata,
deterministic} (\dfa)\index{\dfa} or nondeterministic\index{automata,
nondeterministic} (\nfa).\index{\nfa} A \dfa\ example is presented
first.

\subsection{\bf State-transition Diagram} 

One can draw an intuitive diagram representing a \dfa.  
The diagram below recognizes properly rounded values approximating $2/3$.  
The value to be recognized must start with zero, then a dot, 
then any number of sixes and terminate with a seven.  
Each state is boxed; the initial state is labelled S; 
the final state is double boxed.

\stepcounter{figure}				% simulate /figure
\newcounter{fsm}[chapter]
\setcounter{fsm}{\value{figure}}
% a state diagram for 0.6*7
\begin{picture}(200,60)(-60,10)

\put(30,20){\fbox{S}}
\put(44,22){\vector(1,0){26}}
\put(50,25){$0$}

\put(70,20){\fbox{A}}
\put(84,22){\vector(1,0){26}}
\put(90,25){\bf.}

\put(110,22){\fbox{B}}

\put(124,26){\line(1,0){5}}
\put(129,26){\line(0,1){20}}
\put(129,46){\line(-1,0){30}}
\put(99,46){\line(0,-1){20}}
\put(99,26){\vector(1,0){11}}
\put(117,37){$6$}

\put(124,22){\line(1,0){5}}
\put(129,22){\line(0,-1){7}}
\put(129,15){\vector(1,0){21}}
\put(135,19){$7$}

\put(150,12){\fbox{\fbox{C}}}
\end{picture}
\begin{center}
Figure \thefigure. A \dfa\ for rounded values of $2/3$.
\end{center}

The same information can be represented as a {\em transition matrix}, 
with current state and next input symbol as coordinates, 
the successor state is at the intersection, 
blank means error(reject).  Final states are boxed.

\stepcounter{figure}				% simulate /figure
\newcounter{famatrix}[chapter]
\setcounter{famatrix}{\value{figure}}
% a transition matrix for 0.6*7
\begin{picture}(200,90)(-90,0)
\put(50,80) {\em see}
\put(-20,30){\em in}
\put(-30,20){\em state}
\put(20,70){0}
\put(43,70){\bf .}
\put(60,70){6}
\put(80,70){7}
\put(-3,51){\fbox{C}}
\put( 0,36){B}
\put(60,36){B}
\put(80,36){C}
\put( 0,19){A}
\put(40,19){B}
\put(0 , 2){S}
\put(20, 2){A}
\put(15,66){\line(1,0){80}}     % horizontal lines
\put(15,49){\line(1,0){80}}     % 17 points per row
\put(15,32){\line(1,0){80}}
\put(15,15){\line(1,0){80}}
\put(15,-2){\line(1,0){80}}
\put(15,-2){\line(0,1){68}}     % vertical lines
\put(35,-2){\line(0,1){68}}
\put(55,-2){\line(0,1){68}}
\put(75,-2){\line(0,1){68}}
\put(95,-2){\line(0,1){68}}

\end{picture}

\begin{center}
Figure \thefigure. Transition matrix for rounded values of $2/3$.
\end{center}


\subsubsection{Exercises}
\begin{enumerate}\setcounter{enumi}{\value{RunningExercise}}

\item  Draw a diagram for a \dfa\ that recognizes any sequence of
nickels and dimes (N and D) that adds up to a quarter. 
(Hint: let state {\bf k} represent an accumulation of $5k$ cents.)

\item Draw a diagram for an automaton that recognizes a sequence
of zero or more a's, followed by a sequence of zero or more b's,
followed by one c.

\item 
Draw a diagram and transition matrix for a \dfa\ that recognizes
positive integers.\footnote{The statement of a recognition condition
for an automaton implies in addition ``and rejects anything else.''}

\item Draw a diagram or transition matrix for a \dfa\ that recognizes 
rounded values of $1/7$.

\item  Suppose that you have a \dfa\ that recognizes truncated
representations of fraction $1/n$.  
How can you transform it into a \dfa\ that 
recognizes rounded representations of $1/n$?  
(Hint: does your solution work for $1/101$?).

\newcounter{abcexercise}
\setcounter{abcexercise}{\value{enumi}}

\setcounter{RunningExercise}{\value{enumi}}
\end{enumerate}

If any entry in the transistion matrix has more than one value,
or has the empty string as a value, the \fa\ is a \nfa.  

% - - - - - - - - - - - - - - - - - - - - - - - - - - - - - - - - - - 
\subsection{\cfg\ for Automata}

\stepcounter{table}					% simulate \table
\newcounter{FAschema}[chapter]
\setcounter{FAschema}{\value{table}}

A \fa\ can also be defined by a \cfg.  
If the rules in $\Pi$ are restricted to one of the 
three forms shown in Table~\thetable\ the grammar is a \fa.  
The phrase names correspond to states; the rules define transitions; 
the goals are the final states (such a \cfg\ is
sometimes called a {\em left linear grammar}).


\begin{samepage}
\begin{eqnarray}
	A &\rightarrow& B {\rm a}		\\
	A &\rightarrow& B				    \\
	A &\rightarrow& \lambda		
\end{eqnarray}
\begin{center}
Table~\thetable: Schema for Finite Automata\index{schema, finite automata}
\newcounter{NfaDfa}[chapter]
\setcounter{NfaDfa}{\value{table}}
\end{center}
\end{samepage}

\noindent More formally a \cfg,

\begin{center}$\langle V_I, V_N, V_G, \Pi\rangle$ \end{center}

\noindent
is a \fa\ if

\begin{center}
   $\Pi \subseteqdef V_N\times\{V_N\circ V_I \cup V_N \cup \{\lambda\}\}$
\end{center}

\noindent
The set

\begin{center}$V_S \eqldef \Pi\Rstrict\{\lambda\}$\end{center}

\noindent
defines the {\em start states}; $V_G$ is the set of final states.

The first kind of \fa\ rule defines the
transitions (often called {\em shifts}).\index{shift} 
The shifts are deterministic if
\begin{displaymath}
\forall A{\rm a}.\ {\rm size}(\Pi\Rstrict\{A\rm a\}) \leq 1
\end{displaymath}
That is to say, for no phrase name $A$ is there more than one shift
defined for any input symbol a.

The second kind of rule is called an empty
transition (from $B$ to $A$).\index{transition, empty}\footnote{An
empty transition is often called an $\epsilon$
transition\index{transition, epsilon} in literature using letter
$\epsilon$ to denote the empty string.}  
The third kind of rule provides a start state.

For example, the \cfg\ for rounded values of 2/3 is:
\begin{tt}
\begin{tabbing}
xxxxxxxxxxxxxxxxxxxxxxxxx\=xxx\=\kill
\>C = B \sq 7\sq;                 \\
\>B = B \sq 6\sq;                 \\
\>B = A \sq.\sq;                  \\
\>A = S \sq 0\sq;                 \\
\>S = ;
\end{tabbing}
\end{tt}
\begin{center}
\cfg\ Form of a \dfa\ Describing 2/3
\end{center}

If the shifts are deterministic and there are no empty transitions,
and there is only one start state, then the automaton is a \dfa;
otherwise it is an \nfa.\index{automaton, deterministic} 

If one uses a \fa\ \cfg\ to reduce a string, one starts with the input text.
The only reductions that can be applied are those for start states.  
At each subsequent stage the \fa\ transitions according to
one of the other rules for 2/3 until the \fa\ stops.  
The derivation of the S/R sequence using the \sform\ predicate 
(from ContextFreeGrammar) follows.  
As usual it is read bottom to top (reducing).

\begin{tt}
\begin{tabbing}
xxxxxxxxxxxxxxxxxxxxxxx\=xxxxxxxxxxxxxxx\=\kill
\>{\sform(C, $\lambda$)}    \>$_{\rm reduce}$             \\
\>{\sform(B7, $\lambda$)}   \>$_{\rm shift}$              \\
\>{\sform(B, 7)}            \>$_{\rm reduce}$             \\
\>{\sform(B6, 7)}           \>$_{\rm shift}$              \\
\>{\sform(B, 67)}           \>$_{\rm reduce}$             \\
\>{\sform(B6, 67)}          \>$_{\rm shift}$              \\
\>{\sform(B, 667)}          \>$_{\rm reduce}$             \\
\>{\sform(A., 667)}         \>$_{\rm shift}$              \\
\>{\sform(A, .667)}         \>$_{\rm reduce}$             \\
\>{\sform(S0, .667)}        \>$_{\rm shift}$              \\
\>{\sform(S, 0.667)}        \>$_{\rm reduce}$             \\
\>{\sform($\lambda$,0.667)} \>$_{\rm start}$
\end{tabbing}
\end{tt}
\begin{center}
\cfg\ proof of $0.667 \in {\cal L}(2/3)$
\end{center}

\subsubsection{Exercises}
\begin{enumerate}\setcounter{enumi}{\value{RunningExercise}}

\item Write down the grammars defining the automata derived in
the previous set of exercises. 

\item Write a program to execute the \dfa\ in Figure~\thefsm.
\index{\dfa, implementation}

\item  Write a program to execute an arbitrary \dfa.  
(Hint: represent $\Pi$ as a transition matrix.)

\item Write a program to execute an arbitrary \nfa.  (Hint: use
threading or backtracking.)\index{\nfa, implementation}

\item  Show how to derive a grammar $\cal A'$ for the {\em sequence of
states} passed to accept a string from the grammar $\cal A$ defining the
\fa. (Hint $V_I' = V_N$).

\newcounter{TransitionExercise}
\setcounter{TransitionExercise}{\value{enumi}}
\setcounter{RunningExercise}{\value{enumi}}
\end{enumerate}

% - - - - - - - - - - - - - - - - - - - - - - - - - - - - - - - - - - 
\subsection{\nfa\ to \dfa}

The surprising fact is that any (bad) \nfa\ can be systematically
transformed into a (good) \dfa.

One can transform an \nfa\ into a \dfa\ a little bit at a time.  
The idea is to use \cfg\ transformations, each removing some
non-determinancy, while preserving the language.

If a \nfa\ has a rule with an empty right-hand side (which is
forbidden in a \dfa), one can {\bf remove} the rule. Any immediate
circularity can be removed whenever it occurs or is introduced.  
Here is an example with A as the start state and C as a final state.
The problem is an empty transition from A to B.  

\begin{samepage}
\begin{tt}
\begin{tabbing}
xxxxxxxxxxxxxxx\=xxx\=\kill
\>C = B \sq 1\sq;                 \\
\>B = A;                          \\
\>B = A \sq 2\sq;                 \\
\>A = ;                           \\
\end{tabbing}
\end{tt}
\end{samepage}

\noindent Following the removal of the "bad" rule 
 and adding a new rule substituting A where B was used:

\begin{samepage}
\begin{tt}
\begin{tabbing}
xxxxxxxxxxxxxxx\=xxx\=\kill
\>C = B \sq 1\sq;                 \\
\>C = A \sq 1\sq;                 \\
\>B = A \sq 2\sq;                 \\
\>A = ;                           \\
\end{tabbing}
\end{tt}
\end{samepage}

\noindent 
This new \cfg\ is a \dfa.  
It can be used to parse the string '21' as follows:

\begin{tt}
\begin{tabbing}
xxxxxxxxxxxxxxx\=\kill
\>\sform(C, $\lambda$)            \\
\>\sform(B1, $\lambda$)           \\
\>\sform(B,  1)                   \\
\>\sform(A2, 1)                   \\
\>\sform(A, 21)                   \\
\>\sform($\lambda$, 21)
\end{tabbing}
\end{tt}

Here is an example with both kinds of non-determinancy.
The \cfg\ has an erasing rule and also
state A has two different transitions on 1.  
The language is \{101, 110, 111\}.

\begin{samepage}
\begin{tt}
\begin{tabbing}
xxxxxxxxxxxxxxx\=xxx\=\kill
\>F = D \sq 0\sq;                 \\
\>F = E \sq 1\sq;                 \\
\>E = D;                          \\
\>E = B \sq 0\sq;                 \\
\>D = C \sq 1\sq;                 \\
\>C = A \sq 1\sq;                 \\
\>B = A \sq 1\sq;                 \\
\>A = ;                           \\
\end{tabbing}
\end{tt}
\end{samepage}

\noindent After eliminating the empty transition (as before), 
we have the transformed \cfg:

\begin{samepage}
\begin{tt}
\begin{tabbing}
xxxxxxxxxxxxxxx\=xxx\=\kill
\>F = D \sq 0\sq;                 \\
\>F = E \sq 1\sq;                 \\
\>F = D \sq 1\sq;                 \\
\>E = B \sq 0\sq;                 \\
\>D = C \sq 1\sq;                 \\
\>C = A \sq 1\sq;                 \\
\>B = A \sq 1\sq;                 \\
\>A = ;                           \\
\end{tabbing}
\end{tt}
\end{samepage}


\noindent After fusing the two ambiguous rules, adding a new symbol X, 
and using X where C and B had been used,
we have the transformed \cfg\ for the \dfa:

\begin{samepage}
\begin{tt}
\begin{tabbing}
xxxxxxxxxxxxxxx\=xxx\=\kill
\>F = D \sq 0\sq;                 \\
\>F = D \sq 1\sq;                 \\
\>F = E \sq 1\sq;                 \\
\>E = X \sq 0\sq;                 \\
\>D = X \sq 1\sq;                 \\
\>X = A \sq 1\sq;                 \\
\>A = ;                           \\
\end{tabbing}
\end{tt}
\end{samepage}

\stepcounter{figure}				% simulate /figure
\newcounter{dfamatrix}[chapter]
\setcounter{dfamatrix}{\value{figure}}
\noindent The corresponding transition diagram is given in Figure~\thefigure.

\begin{center}
\begin{picture}(220,110)(-60,-20)
\put(25,80) {\em see}
\put(-20,20){\em in}
\put(-30,10){\em state}
\put(20,70){0}
\put(43,70){1}
\put(-3,51){\fbox{F}}
\put( 0,36){E}
\put(40,36){F}
\put( 0,19){D}
\put(20,19){F}
\put(40,19){F}
\put( 0,2){X}
\put(20,2){E}
\put(40,2){D}
\put( 0,-15){A}
\put(40,-15){X}
\put(15,66){\line(1,0){40}}     % horizontal lines
\put(15,49){\line(1,0){40}}     % 17 points per row
\put(15,32){\line(1,0){40}}
\put(15,15){\line(1,0){40}}
\put(15,-2){\line(1,0){40}}
\put(15,-19){\line(1,0){40}}
\put(15,-19){\line(0,1){85}}     % vertical lines
\put(35,-19){\line(0,1){85}}
\put(55,-19){\line(0,1){85}}

\end{picture}
\end{center}

\begin{center}
Figure \thefigure. Transition matrix for constructed \dfa.
\end{center}


\subsection{Big Bang \nfa\ to \dfa\ transformation}

There is a ``big-bang'' algorithm to do the \nfa-\dfa\ transformation all in
one step.  See, for example, Wikipedia on "Powerset construction."
The underlying concept is collecting all states with the same effect into sets,
then defining a new machine based on sets of the original states.

The algorithm can be expressed in terms of the \nfa\ \cfg.  
Suppose $\cal G$ is the \nfa\ and
${\rm A}\in V_N, {\rm B}\in V_N, {\rm b}\in V_I$. 
The sought-after \dfa\ $\cal G'$ is recursively defined as follows:
\begin{eqnarray*}
  {\cal M}({\rm B}, {\rm b}) 
     &\eqldef& \{{\rm A}\;|\;{\rm A}\leftarrow^*{\rm Bb}\}    \\
  {\cal E} 
     &\eqldef& \{{\rm A}\;|\; {\rm A}\leftarrow^*\lambda\}    \\
  V_I'  
     &\eqldef& V_I                                            \\
  V_N'     
     &\eqldef& 
       \{{\cal E}\} \cup
       \{ {\cal M}({\rm B},{\rm b})\}                         \\ 
  V_G'
     &\eqldef& \{V_G\}                                        \\
  \Pi'     
     &\eqldef& \{{\cal E}\leftarrow\lambda\}                  \\
     &\cup& 
       \{{\rm A}'\leftarrow{\rm B}'{\rm b} \;|\; 
         {\rm B}'\in{\cal D}(\Pi') 
       \wedge {\rm A}'={\cal M}({\rm B},{\rm b})
       \wedge {\rm B}\in {\rm B}'\}
\end{eqnarray*}

\end{document}