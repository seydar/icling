% FILE:		    RegularExpressions.tex
% PURPOSE:  	The definition of regular expressions
% COPYRIGHT:  W.M.McKeeman 2007


\input xhead

\begin{document}

\begin{center}
\begin{small}
\noindent --contents of Regular Expressions--
\begin{quote}
\raggedright
Rationale\\
Definitions\\
\end{quote}
--requires--
\begin{quote}
\raggedright
Notation Supporting Grammars\\
\end{quote}
\end{small}
\end{center}

\vspace{1em}


\begin{center}
    {\large \bf Regular Expressions} 
\end{center}

\vspace{1em}

Regular expressions serve the same purpose as \cfg{s}: 
they syntactically define strings of symbols.  
One advantage is conciseness and ease of implementation.  
One disadvantage is that they can describe fewer things than \cfg{s}.  
Three issues favor a discussion of regular expressions.  

The first issue is that regular expression expressiveness can be mixed 
in with \cfg{s}.  The new grammars are called regular expression grammars
(\reg{s}).  See the discussion on Regular Expression Grammars.
The second issue is that \reg{s} are an essential step in building 
recursive parsers.  See the discussion on Top-down Parsers.
The third issue is a close relation between regular expressions 
and finite automata.
Finite automata, in turn, form the basis for automatic parsers.  

{\bf Note:} This is {\em not} a discussion about the glob form of regular
expressions used in command-line utilites such as
\begin{verbatim}
  ls *.[ch]*
\end{verbatim}

\vspace{1em}
\noindent {\bf Definitions}

A regular expression is a combination of input symbols ($V_I$) and
meta-symbols.  
The meta-symbols are the following meta-operators:
\vspace{2ex}

\stepcounter{table}					% simulate \table
\newcounter{RegOps}[chapter]
\setcounter{RegOps}{\value{table}}
\begin{samepage}
\begin{tabbing}
xxxxxxxxxxxxxxx\=xxx\=\kill
\>$()$                      \> \mbox{meta-parentheses}              \\
\>$\circ$                   \> \mbox{regular expression `catenate'} \\
\>$|$                       \> \mbox{regular expression `or'}       \\
\>\&                        \> \mbox{regular expression `and'}      \\
\>$-$                       \> \mbox{regular expression difference} \\
\>$^*$                      \> \mbox{zero or more repetitions}      \\
\>$^{\scriptscriptstyle+}$  \> \mbox{one or more repetitions}       \\
\>$^?$                      \> \mbox{zero or one (optional item)}   \\
\>$^n$                      \> \mbox{exactly $n$ repetitions)}      \\
\>$\tw$                     \> \mbox{regular expression complement} \\
\\
\>Table~\thetable: Regular Expression Operators

\end{tabbing}
\end{samepage}

\begin{table}
\begin{eqnarray*}
{\cal L}(\lambda) &\eqldef& \{\lambda\}                                 \\
{\cal L}({\rm a}) &\eqldef& \{{\rm a}\}                                 \\
{\cal L}((A))     &\eqldef& {\cal L}(A)                                 \\
{\cal L}(A B)   &\eqldef& {\cal L}(A)\circ{\cal L}(B)                   \\
{\cal L}(A|B)   &\eqldef& {\cal L}(A) \cup {\cal L}(B)                  \\
{\cal L}(A\&B)  &\eqldef& {\cal L}(A) \cap {\cal L}(B)                  \\
{\cal L}(A-B)&\eqldef& {\cal L}(A) - {\cal L}(B)                        \\
{\cal L}(\mbox{\tw}A)&\eqldef& V_I^* - {\cal L}(A)		                \\
\nonumber && \\
{\cal L}(A^0)   &\eqldef& \{\lambda\}                                   \\
{\cal L}(A^i)   &\eqldef& {\cal L}(A^{i-1}A)                            \\
{\cal L}(A^k_j) &\eqldef& \bigcup_{i=j}^k {\cal L}(A^i)	                \\
{\cal L}(A^*)   &\eqldef& {\cal L}(A^\infty_0)                          \\
{\cal L}(A^{\scriptscriptstyle+})&\eqldef& {\cal L}(A^\infty_1)         \\
{\cal L}(A^?)   &\eqldef& {\cal L}(A^1_0)
\end{eqnarray*}
\caption{Regular Expression Languages}
\end{table}

\noindent 
When the meaning is clear from context, the catenation operator
$a \circ b$ is often implied by the juxtaposition $ a\ b$.  
Applied to sets $A \circ B$ means the catenation of all pairs,
the first taken from $A$ and the second from $B$.

Let a $\in V_I$ and $A$, $B$ designate regular expressions.  
Given a regular expression $A$, the set of strings,
${\cal L}(A)$, defined by it can be built up by repetitive application
of the rules in Table~\thetable.  

The regular expression operators $\&$, \tw and $-$ are useful for 
describing things but are harder to implement for recognizing things,
so they are often omitted.

\subsubsection{Exercises}
\begin{enumerate}\setcounter{enumi}{\value{RunningExercise}}
\item What are
${\cal L}$($\tw()$),
${\cal L}$(a$|$b),
${\cal L}$(a$|$a),
${\cal L}$(a\&b),
${\cal L}$(a\&a),
${\cal L}$(a$^*$),
${\cal L}$(a$\superplus$),
${\cal L}$(a$^9$),
${\cal L}$(a$^9_3$), and
${\cal L}$(cr$^?$a(zy$|$t))?

\item Using Letter, Digit and Any, describe the following \tname{c} 
programming language constructs as regular expressions.
\begin{enumerate}
\item identifier
\item real number literal (e.g. 0.314E+1)
\item integer constant (e.g 123)
\item all X reserved words  (e.g. if, fi...)
\item X comment
\item C comment (hard)
\end{enumerate}

\item The body of a telegram is a sequence of words.  
It is also a sequence of lines of at most 80 characters.  
Write a REG using $\&$, and phrase names tele, line, word, alnum and ch
that helps define the telegram layout.

\setcounter{RunningExercise}{\value{enumi}}

\end{enumerate}


\end{document}