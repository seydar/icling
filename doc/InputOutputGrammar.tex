% FILE:		    InputOutputGrammar.tex
% PURPOSE:  	The theory and use of input-output grammars
% COPYRIGHT:  W.M.McKeeman 2007

\input xhead

\begin{document}

\begin{center}
\begin{small}
\noindent --contents of Input-Output Grammars--
\begin{quote}
\raggedright
Definition\\
Proposition Grammar\\
Interpretation\\
Application\\
Self-description\\
\end{quote}
--requires--
\begin{quote}
\raggedright
Context-free Grammars\\
Regular Expressions\\
Regular Expression Grammars\\
\end{quote}
\end{small}
\end{center}

\vspace{1em}

\noindent {\large \bf Input-Output Grammars} 
\vspace{1em}

\subsubsection{Definition of Input-Output Grammar}

One last form of grammar needs to be defined.  The new grammar, named
an Input/Output grammar (\iog), 
has a new set of symbols for output.\footnote{The 
output symbols are analogous to the \tname{yacc} actions.}
Extending the formalisms for context-free grammars, we have:

\begin{samepage}
\begin{equation}
{\cal G} \eqldef \langle V_I, V_O, V_N, V_G, \Pi\rangle
\label{IOGrammarDefinition}
\end{equation}
\index{$\cal G$}\index{$V_I$}\index{$V_O$}\index{$V_N$}
\index{$V_G$}\index{$\Pi$}

\noindent
consisting of a set of input symbols, a set of output symbols, a set
of phrase names, a set of goals, and a set of reductions.  It
satisfies the following constraints:

\begin{eqnarray*}
     V                &\eqldef& V_I \cup V_O \cup V_N    \\
     V_I \cap V_O     &\eqldef& \{\}				      \\
     V_I \cap V_N     &\eqldef& \{\}				      \\
     V_O \cap V_N     &\eqldef& \{\}				      \\
     V_G	            &\subseteqdef& V_N			    \\
     \Pi	            &\subseteqdef& V_N\times V^*
\end{eqnarray*}
\end{samepage}


\noindent
There is no abstract way to distinguish $V_I$ and $V_O$.  
In fact in some uses they will be interchanged.  
The \cfg\ is just an \iog\ with $V_O=\{\}$.  

\subsubsection{The Proposition Grammar (again)}

\begin{samepage}
\begin{em}
\begin{tabbing}
xxxxxxxxxxxxxxxxxxx\=Propositionxx\=xxx\=\kill
\>Proposition \>\la\> Disjunction                      {\rm 0}  \\        
\>Disjunction \>\la\> Disjunction $\vee$ Conjunction   {\rm 1}  \\
\>Disjunction \>\la\> Conjunction                      {\rm 2}  \\
\>Conjunction \>\la\> Conjunction $\wedge$ Negation    {\rm 3}  \\
\>Conjunction \>\la\> Negation                         {\rm 4}  \\
\>Negation    \>\la\> $\neg$ Boolean                   {\rm 5}  \\
\>Negation    \>\la\> Boolean                          {\rm 6}  \\
\>Boolean     \>\la\> {\rm t}                          {\rm 7}  \\
\>Boolean     \>\la\> {\rm f}                          {\rm 8}  \\
\>Boolean     \>\la\> {\rm(} Disjunction {\rm)}        {\rm 9}
\end{tabbing}
\end{em}

\begin{center}
Table~\thetable: Proposition Input-Output Grammar
\end{center}
\end{samepage}


\subsubsection{Interpretation of Input-Output Grammar}

Left-to-right order is required by the following 
alternative definition; 
in this case it will not only consume the input $\tau$, 
but also construct the output $\rho$.
The predicate \sform\ is used, as in context-free grammars,
to prove a string is in the language defined by the grammar.
It is sufficient for the purposes here to restrict the \iog\
to rules with the output symbol at the end of each rule.

\stepcounter{table}					% simulate \table
\newcounter{reductions}[chapter]
\setcounter{reductions}{\value{table}}
\begin{eqnarray*}
  G \in V_G &\Rightarrow& {\sform}(G, \lambda, \lambda)          \\
  B\la\beta{\rm r}  \in \Pi \wedge {\sform}(\sigma B, \tau, \rho) 
      &\Rightarrow& {\sform}(\sigma\beta, \tau, {\rm r}\rho)            \\
  {\rm a} \in V_I \wedge {\sform}(\sigma {\rm a}, \tau, \rho)
      &\Rightarrow& {\sform}(\sigma, {\rm a}\tau, \rho)           \\
  {\sform}(\lambda, \tau, \rho) &\Rightarrow& G {\stackrel{\rho}{\la}} \tau
\end{eqnarray*}
\begin{center}
Table~\thetable: Left-to-right Reduction Application
\end{center}

\noindent The first implication defines the goal.  
The second implication applies a reduction on the top of the parse stack.  
The third implication shifts a symbol 
from the input text and pushes it on the top of the stack.  
The {\em parse} is the third argument to $\sform$.

\subsubsection{Application of Input-Output Grammar}

Repeating the parse of ${\rm f}\vee{\rm t}\wedge\neg{\rm f}$:

\stepcounter{table}					% simulate \table
\newcounter{IOGProof}[chapter]
\setcounter{IOGProof}{\value{table}}
\begin{tabbing}{ll}
\quad\quad\quad\quad\=
xxxxxxxxxxxxxxxxxxxxxxxxxxxxxxxxxxxxxxxxxxxxxxx\=\kill
\>$\sform(Proposition,\lambda, \lambda)$                               \\
\>$\sform(Disjunction,\lambda, 0)$                                     \\
\>$\sform(Disjunction\vee Conjunction, \lambda, 10)$                   \\
\>$\sform(Disjunction\vee Conjunction \wedge Negation, \lambda, 310)$  \\
\>$\sform(Disjunction\vee Conjunction\wedge\neg Boolean, \lambda, 5310)$\\
\>$\sform(Disjunction\vee Conjunction\wedge\neg\rm f, \lambda, 85310)$ \\
\>$\sform(Disjunction\vee Conjunction\wedge\neg,\rm f, 85310)$         \\
\>$\sform(Disjunction\vee Conjunction\wedge,\neg\rm f, 85310)$         \\
\>$\sform(Disjunction\vee Conjunction,\wedge\neg\rm f, 85310)$         \\
\>$\sform(Disjunction\vee Negation,\wedge\neg\rm f, 485310)$           \\
\>$\sform(Disjunction\vee Boolean,\wedge\neg\rm f, 6485310)$           \\
\>$\sform(Disjunction\vee\rm t,\wedge\neg f, 76485310)$                \\
\>$\sform(Disjunction\vee,\rm t\wedge\neg f, 76485310)$                \\
\>$\sform(Disjunction,\vee\rm t\wedge\neg f, 76485310)$                \\
\>$\sform(Conjunction,\vee\rm t\wedge\neg f, 276485310)$               \\
\>$\sform(Negation,\vee\rm t\wedge\neg f, 4276485310)$                 \\
\>$\sform(Boolean,\vee\rm t\wedge\neg f, 64276485310)$                 \\
\>$\sform(\rm f,\vee t\wedge\neg f, 864276485310)$                     \\
\>$\sform(\rm\lambda,f\vee t \wedge\neg f, 864276485310)$
\end{tabbing}
\begin{center}
Table~\thetable: 
Proof of and parse for
$\sq{\rm f}\vee{\rm t}\wedge\neg{\rm f}\sq \in \cal L(G)$
\end{center}


\end{document}