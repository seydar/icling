% FILE:		    ReferenceX.tex
% PURPOSE:  	Reference Manual for X
% COPYRIGHT:  W.M.McKeeman 2007
% 		latex description of X

\input xhead

\begin{document}

\begin{center}
  \begin{small}
    \noindent --contents of Reference Manual for X--
    \begin{quote} \raggedright
      Whitespace\\
      Comments\\
      Names\\
      Types, Constants and Expressions\\
      Assignment, Input and Output\\
      Statements\\
      Programs\\
      Variables, Implicit Name Introduction\\
      Control: Selection and Iteration\\
      Guards: Logical Operators\\
      Subprograms\\
      Assertions\\
      Errors\\
    \end{quote}
  \end{small}
\end{center}

\vspace{1em}

\begin{center}
 {\large \bf Informal Reference Manual for X} 
\end{center}
\vspace{1em}

People, even programmers, can use language with very little 
{\em a priori\/} knowledge.  We all generalize from examples, 
guess intentions, and succeed in communicating long before the 
schools and textbooks round out our native tongue.  

The language used here is named X;  
it is inspired by the language used by Edsger Dijkstra in his book
{\em A Discipline of Programming}.
Do not look for declarations, data structures or explicit input/output.
Otherwise your expectations from other programming languages will be close 
enough.  

Reaching for the elegance of scientific notation, 
Dijkstra typeset his programs, using a variety of mnemonic glyphs.
As a practical matter, compiler input must be readily prepared and modified.  
An \tname{ascii} compromise has been chosen for \tname{x} source input.  
The compiler \tname{xcom} can, upon request, typeset its input program.  
If you extend \tname{xcom}, you should also extend the typesetting.

The input format (not the typeset format) is used for the examples.

\subsubsection{Whitespace}

Except to separate other symbols in X, whitespace symbols are ignored.

\subsubsection{Comments}

Text starting with \bq\ to the end of line is a comment.  A comment is 
equivalent to whitespace.  If pretty printing is used, the comment
body is interpreted as LaTeX and therefore {\em must conform}
to the LaTeX rules for the interpretation of characters.

\subsubsection{Names}

The concept of a {\em name} -- a sequence of letters and digits -- is 
universal in programming languages.  In X some names are reserved and 
the rest may be used for program constructs.

An {\em operator name\/} is similar -- a sequence of operator 
symbols.\footnote{Frank DeRemer (of LALR fame) suggested this idea.}  
Thus {\tt +}, {\tt :=} and {\tt <=} are operator names in \tname{x}.  
The operator name concept provides a handy form for extensions to \tname{x}.
The operator symbols used in operator names in X are

\begin{verbatim}
    & * + - / : < = > ? | ~
\end{verbatim}

\noindent If other symbols are actually used, the list above is 
automatically extended.  Extra whitespace is sometimes required.  
For example
\begin{verbatim}
  x:=-3;
\end{verbatim}
\noindent will cause a compile error because {\tt :=-} is not an operator.  

The rest of the symbols in X stand for themselves.

\subsubsection{Types, Constants and Expressions}

\tname{x} has three data types: logical, integer, and 
real.\footnote{These three data types are needed to express decisions, 
counting and measurement, thus they are available in almost every 
programming language.}  
Each type has manifest constants (e.g. {\tt true}, {\tt 1}, {\tt 1.0}).  
Integer constants cannot be used where real values are required.

The operators and builtin functions of \tname{x} are given below.  
The signature {\em c(dd)} means the operator accepts two arguments of type 
{\em d} and returns type {\em c}.  
Most numeric operators have two signatures each (polymorphic).  
The letters {\em bir\/} stand for the three X types respectively.

\begin{tt}
\begin{tabbing}
xxxxxx\=xxxxxxxxxxxxxxx\=xxxxxxxxxx\=xxxxxxx\=\kill
\>\em signatures        \>\tt op\>{\em meaning}                       \\
\>\em i(i)  r(r)        \>-x     \>   {\rm minus} x                   \\
\>\em i(ii) r(rr)       \>x+y    \>x  {\rm plus} y                    \\
\>\em i(ii) r(rr)       \>x-y    \>x  {\rm minus} y                   \\
\>\em i(ii) r(rr)       \>x*y    \>x  {\rm times} y                   \\
\>\em i(ii) r(rr)       \>x/y    \>x  {\rm divided by} y              \\ 
\>\em i(ii)             \>x//y   \>   {\rm remainder of} x {\rm divided by} y \\
\>\em b(ii) b(rr)       \>x<y    \>x  {\rm less than} y               \\
\>\em b(ii) b(rr)       \>x<=y   \>x  {\rm less than or equal} y      \\
\>\em b(ii)             \>x=y    \>x  {\rm equal} y                   \\
\>\em b(ii)             \>x\tw=y \>x  {\rm not equal} y               \\
\>\em b(ii) b(rr)       \>x>=y   \>x  {\rm greater than or equal} y   \\
\>\em b(ii) b(rr)       \>x>y    \>x  {\rm greater than} y            \\
\>\em b(b)              \>\tw x  \>   {\rm not} x                     \\
\>\em b(bb)             \>x\&y   \>x  {\rm and} y                     \\
\>\em b(bb)             \>x|y    \>x  {\rm or} y                      \\
\>\em (bb) (ii) (rr)    \>x:=y   \>x  {\rm assigned the value of} y   \\
\>\em i(b)              \>b2i(x) \>   {\rm integer form of logical} x \\
\>\em r(i)              \>i2r(x) \>   {\rm real form of integer} x    \\
\>\em i(r)              \>r2i(x) \>   {\rm integer form of real} x    \\
\>\em r()               \>rand   \>   {\rm random real}
\end{tabbing}
\end{tt}

\noindent 
The type constraints of builtin signatures are used to infer and/or 
check types of operands.  
It is possible to write an \tname{x} program where the type of some 
variables are ill-defined.  
For example, x and y are inferred to have the same type below, 
but that type itself is not yet determined.
\begin{verbatim}
  x := y
\end{verbatim}
This particular type ambiguity can be remedied by inserting 
non-effect arithmetic such as addition of zero.
\begin{verbatim}
  x := y+0
\end{verbatim}
forcing the type to integer.

Type inference provides the capability normally given to declarations in 
other programming languages.

There is one more type reserved for function names.  
Function names can only be assigned, 
passed as parameters or return values, or called.

Arithmetic is native to the hardware: IEEE floating point for
real, 2's complement for integer, and true/false for logical.  
Comparisons for equality between real values are not allowed because 
such a comparison does not make numerical sense.

\subsubsection{Assignment, Input and Output}

If one writes a program consisting of a single assignment:
\begin{tt}
\begin{tabbing}
xxxx\=xxxxxxxxxxxxxxxxxxxxxxxxxxxxxxxxxxx\=\kill
\>h := 1;\>\bq\ this is a comment
\end{tabbing}
\end{tt}

\noindent
one infers that {\tt h} is an integer-valued variable which will be 
assigned the value {\tt 1}.  
This assignment to {\tt h} is apparently wasted, since the value is not 
subsequently used.  
When this program is run, it will present the output

\begin{tt}
\begin{tabbing}
xxxx\=xxxxxxxxxxxxxxxxxxxxxxxxxxxxxxxxxxx\=\kill
\>h := 1
\end{tabbing}
\end{tt}

\noindent
That is, output is achieved in \tname{x} by reporting wasted assignments.  
One can now intuit that undefined variables in the program:

\begin{tt}
\begin{tabbing}
xxxx\=xxxxxxxxxxxxxxxxxxxxxxxxxxxxxxxxxxx\=\kill
\>\verb@b := c | d < 3.1;@  \>\bq input required 
\end{tabbing} 
\end{tt}

\noindent
will require input values for \vname{c} and \vname{d}.  
Indeed, when this program is run, the user will be asked, in turn:

\begin{tt}
\begin{tabbing}
xxxx\=xxxxxxxxxxxxxxxxxxxxxxxxxxxxxxxxxxx\=\kill
\>\tt logical input c := ? \\
\>\tt real input d := ?
\end{tabbing}
\end{tt}

\noindent
Upon receiving a logical value for \vname{c} and a real value for \vname{d}, 
the program will run and report the final value of \vname{b} upon 
completion.

The form of an assignment is a list of variables followed by an equally 
long list of expressions separated by an assignment operator.
The effect is as if all the right-hand side expressions were evaluated, 
and then each simultaneously assigned to its corresponding left-hand side.  
In particular, the assignment

\begin{tt}
\begin{tabbing}
xxxx\=xxxxxxxxxxxxxxxxxxxxxxxxxxxxxxxxxxx\=\kill
\>x,y := y,x
\end{tabbing}
\end{tt}

\noindent
exchanges the values of \vname{x} and \vname{y}.  

The result of assignment where the left-hand side contains a single name 
more than once is not defined.\index{undefined}

\subsubsection{Statements}

There are, in addition to assignments, three other kinds of statements 
in \tname{x}.  Here are some examples of statements:

\vspace{1ex}

\noindent {\em assignment} 

\begin{tt}
\begin{tabbing}
xxxx\=xxxxxxxxxxxxxxxxxxxxxxxxxxxxxxxxxxx\=\kill
\>r, i, b := 1.0, i+2, true
\end{tabbing}
\end{tt}

\noindent {\em selection}

\begin{tt}
\begin{tabbing}
xxxx\=xxxxxxxxxxxxxxxxxxxxxxxxxxxxxxxxxxx\=\kill
\>if x < y  ? x := y-1  \\
\>::\ x = y ?           \\
\>::\ x > y ? x := y+1  \\
\>fi
\end{tabbing}
\end{tt}

\noindent {\em iteration} 

\begin{tt}
\begin{tabbing}
xxxx\=xx\=xxxxxxxxxxxxxxxxxxxxxxxxxxxxxxxxx\=\kill
\>do x < y ? x := x+7.111  \\
\>od                       \\
\end{tabbing}
\end{tt}

\noindent {\em calling a subprogram}

\begin{tt}
\begin{tabbing}
xxxx\=xxxxxxxxxxxxxxxxxxxxxxxxxxxxxxxxxxx\=\kill
\>r, y := myfun := 3.14, 37, false
\end{tabbing}
\end{tt}

\subsubsection{Programs}

Where one statement can appear, a list of statements separated by 
semicolons can appear.  
Statements are executed in the order of the list containing them.  
A statement list can be empty; a trailing semicolon never causes trouble.

A program is a list of statements.  The meaning of a program is its 
input/output mapping; the effect of running a program is to display an 
instance of that input/output mapping.  
All input is collected before program execution and no output is 
reported until the completion of execution.

The input requests and output reports appear in the order in which the 
variables appear in the program.  The runtime system must supply sufficient 
information to make the input/output actions unambiguous
--- associating the name of the variable and its value is sufficient.
(The wise programmer will pick mnemonic names.)

The instantaneous state of a program is a set of variable-value pairs.
The values of variables change during execution.  
The dynamic sequence of changes is, by intention, invisible to the user.  
The initial state is provided by input; the final state is reflected as 
output; what happens in between is the compiler writer's private business.

\subsubsection{Variables, Implicit Name Introduction}

A name appearing in an X program has only one meaning.  It must be reserved, 
or name a subprogram, or name a variable.  The attributes of each name
are derived from the context surrounding its uses.
Any program variable appearing only in an expression is an 
{\em input variable}; 
one appearing only on the left of an assignment is an {\em output variable}. 
If a variable is used before it is assigned, 
the value is undefined.\index{undefined}

\subsubsection{Control: Selection and Iteration}

There are two control constructs in \tname{x}, selection (\mbox{\tt
if-fi}) and iteration (\mbox{\tt do-od}).  
Within \mbox{\tt if-fi} brackets there is a set of alternatives.  
Each alternative consists of a guarding logical expression followed by a 
statement list.

The meaning of a selection is the meaning of the statement list behind 
the first true guard\index{guard} and, if there are no true guards, 
the meaning is {\em abort}. 
The order of tests means that the final guard can be {\tt true}
which then behaves as {\tt else} in other languages.

Except for the delimiters, an iteration has the same form as a selection, 
and also executes the statement list behind the first true guard.  
If there is no true guard, the iteration terminates, otherwise it repeats.

Dijkstra chose nondeterministic, as contrasted to first true value, 
for selection and iteration.  
That choice makes more sense for program generation than it does for
programming.

\subsubsection{Guards: Logical Operators}

The guards are logical expressions, combining relations over numeric 
expressions and the constants {\tt true} and {\tt false} with logical 
operators.  
It is in the nature of \tname{x} that guards cannot change the state of 
the program since there are no side effects (except for builtin {\tt rand}).

\subsubsection{Subprograms}

Any program also defines a subprogram.  In the calling syntax, 
the {\em actual input\/} expressions are on the right; 
the {\em actual output\/} variables are on the left.  
A subprogram named {\tt xyz} is defined in file {\tt xyz.x}.  
For example, if {\tt xyz.x} has two outputs and three inputs,
the following line of \tname{x} code will call it.
\begin{tt}
\begin{tabbing}
xxxx\=xxxxxxxxxxxxxxxxxxxxxxxxxxxxxxxxxxx\=\kill
\>\verb@a,b := xyz := a+1.0, true, 13;@  \>\bq call xyz 
\end{tabbing} 
\end{tt}
\noindent 
The call must supply the expected number and type of inputs; 
the number and type of outputs must correspond to the left-side of the 
call syntax.

The function identifier can be a function type variable.  
Such a variable must have been assigned a literal function name,
either directly or indirectly.

Any subprogram can be run standalone; 
it will request values for its inputs and report its outputs.  
All files that mutually reference each other must be compiled and run 
together.  Recursion is allowed.

\subsubsection{Assertions}

In \tname{c} one can write {\tt assert(expr)} to insure that the {\tt expr} 
evaluates to {\tt true} every time the assertion is executed.
Programmers use this construct to document and automatically check the 
conditions under which this code should work long after the original 
author has turned attention elsewhere.  In \tname{x} one writes

{\tt if expr ? fi;\quad\quad     \bq\ reason for check} 

\noindent to achieve the same effect.  
If the {\tt expr} evaluates to {\tt false}, the selection fails and
the \tname{x} program aborts.

\subsection{Errors}

Debugging an \tname{x} program can be difficult. 
If the \tname{x} program is in an infinite loop, 
only killing MATLAB will stop it.
One can then run under emulation to see where the problem is.
Inserting assignments in the \tname{x} program 
and waiting for output after execution is a way to see
what intermediate values were.

\end{document}

